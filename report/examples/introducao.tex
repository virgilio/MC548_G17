Neste primeiro relatório, apresentamos as bases teóricas que serão
utilizadas nos outros quatro subprojetos de implementação
subseqüentes. Sistemas distribuídos necessitam de uma rede para a
comunicação entre os diversos computadores do sistema, por isso,
iniciamos respondendo algumas questões sobre redes na
seção~\ref{sec:questoes}.

Em seguida, apresentamos resumos e pseudo-códigos de três algoritmos
clássicos de Sistemas Distribuídos que serão utilizados no decorrer da
disciplina. Seguindo a ordem cronológica, o algoritmo de ordenação
total de~\cite{lamport78}, juntamente com um exemplo de uso -- um
algoritmo de exclusão mútua --- são apresentados na
seção~\ref{sec:lamport}. Em seguida, na seção~\ref{sec:herman}
discutimos um algoritmo baseado nesse mecanismo, de Herman e
Verjus~\cite{hermanv79}, para ordenação e replicação de
dados. Finalmente, na seção~\ref{sec:chandy}, apresentamos o algoritmo
de Chandy e Lamport~\cite{chandyl85} para obtenção de estados globais
consistentes num sistema distribuído, publicado em 1985.


