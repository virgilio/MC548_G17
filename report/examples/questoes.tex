\subsubsection*{Quantos classes (tipos) de endereço IP existem? Como
  são especificados?}

Há cinco classes distintas de endereços IPv4, que são identificadas
pelos quatro primeiros bits dos trinta e dois que formam um endereço
IPv4. A classe A tem endereços que começam com 0; os da B começam com
10; os da C começam com 110; os da D começam com 1110 e os da E
começam com 1111. A classe D é reservada para grupos de
\emph{multicast} e a classe E é reservada para uso futuro. 

As classes A, B e C dividem seus endereços IP em duas partes: o número
de rede, que identifica cada rede, e o resto, que identifica um
computador dentro de uma rede. Os números de rede têm em cada classe
respectivamente 8, 16 e 24 bits.

Entretanto, vale lembrar que as antigas classe IP hoje estão
essencialmente obsoletas, por seguirem um esquema demasiado
restritivo, incompatível com a expansão da Internet. Em vez disso, a
rede é segmentada em uma hierarquia de sub-redes seguindo um sistema
conhecido como \emph{Classless Inter-Domain Routing} (CIDR). Em CIDR,
cada rede é especificada como um endereço seguido de um número, que
representa o tamanho em bits do número de rede para endereços daquela
rede.

A cada rede corresponde uma máscara de sub-rede, que é um endereço com
os bits do número de rede em 1 e o resto em 0. Dessa forma, dado um
endereço, basta fazer um E bit-a-bit com sua máscara de sub-rede
associada para se descobrir o número da rede. Pode-se considerar que
as antigas classes de redes possuíam então máscaras de sub-rede
implícitas.

CIDR é interessante pois, além de permitir uma maior granularidade na
atribuição de redes IP, possibilita que se agregue várias redes para
formar uma super-rede, o que torna as tabelas de roteamento mais simples.

\subsubsection*{No caso da rede Ethernet, o uso de broadcast aumenta o
  tráfego na rede? Sim ou não? Justifique.}

Os quadros enviados em Ethernet sempre são vistos por todas as
máquinas que compartilham o mesmo enlace, logo o sistema funciona
naturalmente em \emph{broadcast}, sem sobrecarga adicional. Nesse
contexto, \emph{broadcasts} são enviados uma única vez, e todas as
máquinas escutam e aceitam a mensagem. O que diferencia mensagens de
\emph{broadcast} de mensagens ponto-a-ponto normais é o endereço de
destino: existe um endereço MAC especial que indica que uma mensagem é
destinada a todas as máquinas do enlace.

\subsubsection*{ Como se consegue comunicação ponto-a-ponto na
  Ethernet?}

Embora os quadros sejam recebidos por todas as máquinas em uma mesma
rede, uma dada máquina não manterá qualquer mensagem que receber: ela
o fará baseada no endereço de destino do quadro. Cada placa de rede
possui um endereço MAC, idealmente único, que permite que se enderece
um quadro para uma máquina específica da rede.

\subsubsection*{Após fazer login em uma estação UNIX do laboratório do
  IC, como é possível determinar o endereço de broadcast para todas as
  interfaces de rede (máquinas) dessa rede? Especifique os comandos
  UNIX utilizados para obter a informação sobre broadcast para a rede
  e explique porque foram utilizados e como funcionam.}

Basta utilizar o comando:

\noindent \verb!$ ifconfig  | grep 'inet addr:'| grep -v '127.0.0.1' | awk '{print $3}'!

\noindent O ifconfig exibe a configuração de todas as interfaces de
rede dessa máquina. Os outros comandos são utilizados apenas para
filtrar os endereços de \emph{broadcast} e remover a interface de
\emph{loopback}.

\subsubsection*{O seu grupo de projeto realiza broadcasts em um
  laboratório do IC. Os quadros enviados poderão ser recebidos por
  processos no laboratório vizinho? Suponha que cada laboratório é uma
  rede Ethernet e que as redes dos laboratórios são interligadas por
  roteadores.}

Caso seja utilizado o endereço de \emph{broadcast} usual (\emph{direct
  broadcast}), a passagem do \emph{broadcast} para as redes vizinhas
vai depender da configuração da rede dos laboratórios envolvidos (ips
e máscaras envolvidos) e da configuração do roteador para repassar o
broadcast de uma rede para outra. Uma opção garantida para que o
broadcast não extravase a rede é utilizar o endereço de broadcast
especial \verb!255.255.255.255!.

\subsection*{Multicast e Broadcast na Internet}

Aplicações que usam \emph{multicast} na escala da Internet pública são
praticamente inexistentes. Existem vários problemas de natureza
econômica e de segurança com o protocolo e, como o IPv4 classifica seu
suporte como opcional, a maioria dos roteadores da Internet não o
implementam.

Problemas de segurança incluem ataques que usam o IGMP (protocolo que
controla a entrada e saída de clientes em grupos de \emph{multicast}) como
amplificador de pacotes para negação de serviço. Outro problema
relacionado a segurança é que não há como o remetente controlar quais
destinatários são ou não aceitos.

Tráfego \emph{multicast} também é difícil de rastrear e cobrar, motivo
pelo qual provedores de Internet (ISPs) em geral o desabilitam. O
Mbone é uma alternativa gratuita que consiste de uma infraestrutura de
enlaces paralela à da Internet. Por ser paralela, depende de que ISPs
estejam conectados a ela para que seja acessível a usuários
finais. Como isso raramente acontece, seu uso é basicamente restrito a
aplicações científicas e de engenharia.

O broadcast IP além das redes locais sofre de problemas similares de
segurança e controle de uso de banda, sendo por isso desabilitado
pelos provedores de camadas mais altas.

